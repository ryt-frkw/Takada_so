% TeXWorks : pLaTeX (ptex2pdf)

\documentclass[b5,12pt,fleqn,dvipdfmx]{jsarticle}
\textwidth=154mm
\textheight=229mm

\usepackage[top=1cm, bottom=1.5cm, left=1cm, right=1cm]{geometry}
\usepackage{bxpapersize}
\usepackage{amsmath}
\usepackage{emath}

\begin{document}
絶対値の館 \\
問1 \quad $ kを正の実数とする. xの方程式2k|x-2|-|x^3-12x+20|=0 が異なる4つの実数解\alpha , \beta , \gamma , \delta (ただし, \alpha \textless \beta \textless \gamma \textless \delta )をもつとき, 次の問いに答えよ.  $

(1) $ kのとりうる値の範囲を求めよ. $

(2) $ \alpha , \beta , \gamma , \delta のとりうる値の範囲をそれぞれ求めよ. なお, \alpha の範囲に関しては高校数学の範囲では出すことができないため, 解答者が高校生の場合は解答不要とする.$

(3) $ kの関数 y=\beta \gamma \delta のグラフを描け. $

(4) $ f(k)=|\alpha + \beta + \gamma + \delta|とする. 関数y=f(k)のグラフを描け. $

\end{document}